% Preview source code

%% LyX 1.6.4 created this file.  For more info, see http://www.lyx.org/.
%% Do not edit unless you really know what you are doing.
\documentclass[a4paper,english]{paper}
\usepackage[T1]{fontenc}
\usepackage[latin9]{inputenc}
\pagestyle{empty}
\setlength{\parskip}{\medskipamount}
\setlength{\parindent}{0pt}
\usepackage{color}
\usepackage{graphicx}
\usepackage{setspace}

\usepackage{type1cm}
\renewcommand{\rmdefault}{pnc}
\usepackage{helvet}
\usepackage{courier}

\makeatletter

%%%%%%%%%%%%%%%%%%%%%%%%%%%%%% LyX specific LaTeX commands.
\newcommand{\lyxline}[1][1pt]{%
  \par\noindent%
  \rule[.5ex]{\linewidth}{#1}\par}
%% Because html converters don't know tabularnewline
\providecommand{\tabularnewline}{\\}

%%%%%%%%%%%%%%%%%%%%%%%%%%%%%% User specified LaTeX commands.
\usepackage{babel}
\usepackage{apacite}

\makeatother

\usepackage{babel}

\begin{document}

\textbf{\Large }%
\begin{minipage}[t]{1\columnwidth}%

    \sffamily
    \raggedright


\textbf{\Large Key Research Question}{\Large \par}

\noindent \begin{flushleft}
{\Large How can we identify corresponding tracks within different
diffusion MR brain tractographies?} 
\par\end{flushleft}%
\end{minipage}{\Large \par}

\newpage

\textbf{\Large }%
\begin{minipage}[t]{1\columnwidth}%

    \sffamily
    \raggedright

\noindent \begin{flushleft}
\textbf{\Large Requirements to Produce a Tractography}
\par\end{flushleft}{\Large \par}

\begin{itemize}
\item \begin{flushleft} a local diffusion model (e.g. simple tensor, Q-Ball)
\par\end{flushleft}
\item \begin{flushleft} a tractography algorithm (e.g. FACT, Runge-Kutta)
\par\end{flushleft}
\end{itemize}

\noindent \begin{flushleft} The quality and size of resulting tractographies may differ depending
on the choice of these options.
\par\end{flushleft}

\begin{flushleft} For example, based on the same diffusion tensor principal direction vector field and the same sub-grid, 
the FACT algorithm produced $\sim 100,000$ tracks, whereas the Runge-Kutta algorithm of order 2 (RK2) 
produced $\sim 150,000$.
\par\end{flushleft}
%
\end{minipage}{\Large \par}

\newpage

\textbf{\Large }%
\begin{minipage}[t]{1\columnwidth}%

    \sffamily
    \raggedright



\textbf{\Large Comparison of Tractographies}{\Large \par}

\bigskip

Because the datasets are so different in size and number of tracks, 
identifying corresponding tracks is a \emph{hard} problem. 

\end{minipage}{\Large \par}

\newpage

\textbf{\Large }%
\begin{minipage}[t]{1\columnwidth}%
    \sffamily
    \raggedright


\noindent \begin{flushleft}
\textbf{\Large Track Compression}{\Large \par}
\par\end{flushleft}{\Large \par}

\noindent \begin{flushleft}
Tractography datasets are large and algorithms to analyse them need significant computational resources, in
particular long processing times. 
\par\end{flushleft}{\Large \par}

\noindent \begin{flushleft}
We have developed a geometric technique -- \textsc{Apol} --
which reduces the number of points on the tracks of a tractography
by a factor of 8. 
\par\end{flushleft}{\Large \par}

\noindent \begin{flushleft}
More points are retained where the track has high
curvature and fewer are needed where the track has low curvature. The
time performance exceeds that of information theory based techniques
such as \emph{minimum description length} {[}1{]}.%
\par\end{flushleft}{\Large \par}

\end{minipage}{\Large \par}

\newpage

%
\begin{minipage}[t]{1\columnwidth}%
    \sffamily
    \raggedright


\noindent \begin{flushleft}
\textbf{\Large Track Metrics}{\Large \par}
{\Large \par}
\par\end{flushleft}{\Large \par}

\noindent \begin{flushleft}
Metrics are needed to measure that similarity of two tracks, whether
they are in the same or different tractographies. We have used the
mean average minimum distance metric (MAM) {[}2{]}:
{\Large \par}
\par\end{flushleft}{\Large \par}

$$\mathrm{MAM}(P,Q) = \frac{1}{2}\Big[\frac{1}{\#P}\sum_{p\in P}\min_{q\in Q} |p-q| + \frac{1}{\#Q}\sum_{q\in Q}\min_{p\in P} |q-p|\Big]$$%
\end{minipage}



\newpage

%
\begin{minipage}[t]{1\columnwidth}%
    \sffamily
    \raggedright



\noindent \begin{flushleft}

The \emph{minimum distances} between point of P and points of Q are coloured Red.

The \emph{average mimimum distance of P from Q} is the average length of the Red lines.

The \emph{minimum distances} between point of Q and points of P are coloured Blue.

The \emph{average mimimum distance of P from Q} is the average length of the Blue lines.

The \emph{mean average minimum distance between P and Q} is the mean of these two average. It is symmetric.

{\Large \par}
\par\end{flushleft}{\Large \par}

\end{minipage}



\newpage

%
\begin{minipage}[t]{1\columnwidth}%
    \sffamily
    \raggedright

\textbf{\Large Corresponding Tracks}{\Large \par}

\bigskip

We can pick tracks of interest in one brain, and use the MAM metric to
identify corresponding tracks in other brains quickly and accurately.%
\end{minipage}

\newpage

%
\begin{minipage}[t]{1\columnwidth}%
    \sffamily
    \raggedright

\textbf{\Large Colour Map}{\Large \par}

\bigskip

We have coloured the individual tracks in our tractographies in a way which
uses a wider range of colours than conventional colour maps.

\bigskip

We use the projective plane mapping of [3] by assigning
to a track a colour that corresponds to the direction of the middle segment of a 
5-segment approximation of the track.

\bigskip

This can help with recognising structures such as the \emph{corpus callosum}
whose fibers are roughly parallel across the sagittal plane. 

\end{minipage}

\newpage

%
\begin{minipage}[t]{1\columnwidth}%
    \sffamily
    \raggedright

\textbf{\Large Adding Prior Knowledge }{\Large \par}

\bigskip

We can use a range of sources of prior knowledge
\begin{itemize}
\item white matter atlases {[}4{]}
\item using experts to label regions of the brain e.g. {[}5{]}
\item selecting individual tracks 
\end{itemize}
%
\end{minipage}

\newpage

%
\begin{minipage}[t]{1\columnwidth}%
    \sffamily
    \raggedright

\textbf{\Large Conclusion}{\Large \par}

\bigskip

The correspondence problem has been solved in a manner which is supported
by high speed algorithms and high quality visualisation tools.%
\end{minipage}

\newpage

%


\begin{minipage}[t]{1\columnwidth}%
    \sffamily
    \raggedright


\textbf{\Large Software}{\Large \par}

\bigskip

We are developing Python software
\begin{itemize}
\item \textsc{DiPy:} to support diffusion imaging analysis - now available
in open source at \texttt{http://nipy.sourceforge.net/dipy/}

\item \textsc{Fos: }A 3D engine that supports multiple simultaneous visualisations, see
\texttt{http://github.com/Garyfallidis/Fos}
\end{itemize}
%
\end{minipage}

\newpage

%
\begin{minipage}[t]{1\columnwidth}%

    \sffamily
    \raggedright

\textbf{\Large References}{\Large \par}

\begin{itemize}

\item {{[}1{]}} Lee JG, Han J, Whang KY (2007). Trajectory clustering: a partition-and-group framework. Proc. 2007 ACM SIGMOD 

\item {{[}2{]}} Corouge I, Goutthard S, Gerig G (2004). Towards a shape clustering model of white matter fiber bundles using diffusion tensor MRI.
ISBI 2004, 344--347. 

\item {{[}3{]}} Demiralp C, Hughes JF, Laidlaw DH (2009) Coloring 3D Line Fields Using Boy's Real Projective Plane Immersion,
IEEE Transactions on Visualization and Computer Graphics,
15, 1457--1464,

\item {{[}4{]}} Mori S, Crain BJ, Chacko VP, Van Zijl PCM (1999). 
Three-dimensional tracking of axonal projections in the brain by magnetic resonance imaging. 
Annals of Neurology, 45 (2) 265--269.

\item {{[}5{]}} Pittsburgh Brain Connectivity Challenge (2009). \texttt{http://pbc.lrdc.pitt.edu/?q=2009b-home}

\end{itemize}

\end{minipage}


\newpage

\begin{minipage}[t]{1\columnwidth}%

    \sffamily
    \raggedright

Picking individual tracks with the mouse is fun!!!

\end{minipage}



\newpage

\begin{minipage}[t]{1\columnwidth}%

    \sffamily
    \raggedright

Subject 1 (FACT) in \emph{red}: selected tracks in yellow \\

Subject 2 (FACT) in \emph{cyan}: corresponding tracks found \\

\end{minipage}




\newpage

\begin{minipage}[t]{1\columnwidth}%

    \sffamily
    \raggedright

Subject 1 (FACT) in \emph{red}: selected tracks in yellow \\

Subject 2 (FACT) in \emph{cyan}: corresponding tracks found \\

Subject 1 (RK2): in \emph{blue}: corresponding tracks found \\

\end{minipage}



\newpage

\begin{minipage}[t]{1\columnwidth}%

    \sffamily
    \raggedright

All visualised in 3D simultaneously using a modest graphics card

\end{minipage}



\newpage


\begin{minipage}[t]{1\columnwidth}%

    \sffamily
    \raggedright

\textsc{DiPy} -- Diffusion Imaging in Python

\end{minipage}


\end{document}
